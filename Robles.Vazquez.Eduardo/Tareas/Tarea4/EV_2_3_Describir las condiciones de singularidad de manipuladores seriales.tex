\documentclass[11pt,a4paper,oldfontcommands,oneside]{memoir}
\usepackage[utf8]{inputenc}
\usepackage{microtype}
\usepackage[dvips]{graphicx}
\usepackage{xcolor}
\usepackage{times}
\usepackage{graphicx}
\usepackage[spanish]{babel}
\usepackage[
breaklinks=true,colorlinks=true,
%linkcolor=blue,urlcolor=blue,citecolor=blue,% PDF VIEW
linkcolor=black,urlcolor=black,citecolor=black,% PRINT
bookmarks=true,bookmarksopenlevel=2]{hyperref}

\usepackage{geometry}
% PDF VIEW
% \geometry{total={210mm,297mm},
% left=25mm,right=25mm,%
% bindingoffset=0mm, top=25mm,bottom=25mm}
% PRINT
\geometry{total={210mm,297mm},
left=20mm,right=20mm,
bindingoffset=10mm, top=25mm,bottom=25mm}

\OnehalfSpacing
%\linespread{1.3}

%%% CHAPTER'S STYLE
\chapterstyle{bianchi}
%\chapterstyle{ger}
%\chapterstyle{madsen}
%\chapterstyle{ell}
%%% STYLE OF SECTIONS, SUBSECTIONS, AND SUBSUBSECTIONS
\setsecheadstyle{\Large\bfseries\sffamily\raggedright}
\setsubsecheadstyle{\large\bfseries\sffamily\raggedright}
\setsubsubsecheadstyle{\bfseries\sffamily\raggedright}


%%% STYLE OF PAGES NUMBERING
%\pagestyle{companion}\nouppercaseheads 
%\pagestyle{headings}
%\pagestyle{Ruled}
\pagestyle{plain}
\makepagestyle{plain}
\makeevenfoot{plain}{\thepage}{}{}
\makeoddfoot{plain}{}{}{\thepage}
\makeevenhead{plain}{}{}{}
\makeoddhead{plain}{}{}{}


\maxsecnumdepth{subsection} % chapters, sections, and subsections are numbered
\maxtocdepth{subsection} % chapters, sections, and subsections are in the Table of Contents


%%%---%%%---%%%---%%%---%%%---%%%---%%%---%%%---%%%---%%%---%%%---%%%---%%%

\begin{document}

%%%---%%%---%%%---%%%---%%%---%%%---%%%---%%%---%%%---%%%---%%%---%%%---%%%
%   TITLEPAGE
%
%   due to variety of titlepage schemes it is probably better to make titlepage manually
%
%%%---%%%---%%%---%%%---%%%---%%%---%%%---%%%---%%%---%%%---%%%---%%%---%%%
\thispagestyle{empty}

{%%%
\sffamily
\centering
\Large

~\vspace{\fill}
\includegraphics[scale=1]{logo.png} \\
{\huge 
\vspace{4cm}
Describir las condiciones de singularidad de manipuladores seriales
}
\vspace{2.5cm}

{\LARGE
Eduardo Robles Vázquez
}

\vspace{2.5cm}

Universidad Politécnica de la Zona Metropolitana de Guadalajara

\vspace{3.5cm}

Profesor: Carlos Enrique Morán Garabito

\vspace{\fill}

01 de Octubre de 2019

%%%
}%%%

\vspace{.5cm}
\hfill\break




\tableofcontents*

\clearpage

%%%---%%%---%%%---%%%---%%%---%%%---%%%---%%%---%%%---%%%---%%%---%%%---%%%
%%%---%%%---%%%---%%%---%%%---%%%---%%%---%%%---%%%---%%%---%%%---%%%---%%%

\chapter{Singularidad}
Una singularidad es una configuración de un manipulador en serie en el que los parámetros de la articulación ya no definen completamente la posición y orientación del efector final. Las singularidades se producen en configuraciones cuando los ejes de las articulaciones se alinean de una manera que reduce la capacidad del brazo para colocar el efector final.\\

En una singularidad, el efector final pierde uno o más grados de libertad de torsión. Una singularidad generalmente no es un punto aislado en el espacio de trabajo del robot, sino un sub-múltiple.\\

\section{Tipos de Singularidad}
Singularidad arquitectónica: Los robots en serie con menos de seis articulaciones independientes son siempre singulares en el sentido de que nunca pueden abarcar un espacio de giro de seis dimensiones.\\

Singularidad Limite: Se llama singularidad limite cuando el manipulador está completamente distendido o retraído.\\

Singularidad Interna: Se llama singularidad interna cuando ocurre el alineamiento de dos o más ejes de los sistemas de coordenadas, tornando las líneas del Jacobiano linealmente dependientes. Este tipo de singularidad puede ocurrir en cualquier posición del actuador final.\\

\section{Importancia }
Es importante conocer las configuraciones singulares del robot por las siguientes razones: 
\\

1. Causa pérdida de movilidad del robot.
\\

2. Cuando el robot está en una configuración singular, pueden existir infinitas soluciones para la cinemática inversa.
\\

3. Cuando el manipulador se aproxima a una configuración singular, una pequeña velocidad del actuador final provoca grandes velocidades en el accionamiento del robot.


\vspace{2cm}
\hfill

\bibliographystyle{plain}
\bibliography{bibliografia}


\end{document}

